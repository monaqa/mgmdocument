\documentclass{jsarticle}

\usepackage[bLA,func]{mgmmath}

\title{mgmmath.sty test}
\author{最上伸一}

\begin{document}
\maketitle

mgmmath.sty のテスト.

mgmmath.sty の概要を以下に述べる.
\begin{itemize}
\item Basic characteristics:$\fpillar\dif yx$などのコマンドが使える.
\item Sets:$\rset$やら$\nposset,\nzset$などの集合を書きやすくした自由なコマンドを集めたもの.
\item Abbreviation:省略形を集めたもの.$\6$や括弧の省略$\p{x+\dfrac12}$,さらには
\begin{align}
\inti f(x)dx
\end{align}
など,多岐にわたる便利な省略コマンド.
\item LA:オプション.線形代数.pLAオプションも同じ意味.\texttt{\yen vtr}コマンドは,
縦ベクトルを簡単に書きたいときに便利.
\begin{align}
\imat{A}\imat<m,s>{\bm x}=\vtr[1,2,3]
\end{align}
{こういうこと}もできる.

\item bLA:オプション.LAとほぼ同じだが,行列やベクトルを表すコマンドが角括弧になる.
pLA(LA) と bLA を同時に呼び出したときは bLA が優先されるが,
片方のみ呼び出すことを想定して作っている.

\item func:オプション.Fourier 変換$\Fou[f]$や$\sinc$関数,$\div \bm A,\rot \bm B$などが使える.
\end{itemize}

以下,それらの詳細を述べる.
\section{Basic characteristics}
\begin{itemize}
\item 微分演算子を楽に打つコマンド:\texttt{\yen dif[2]\{y\}\{x\},\yen pd[2]\{z\}\{x\}}.
dif は常微分を,pdは偏微分を表示する.
\begin{align}
\dif{y}{x},\;\;\pd[2]{z}{x}
\end{align}
などのように使う.
残念ながら,このコマンドには現在$\dfrac{\6^2z}{\6x\6y}$といった,
異なる変数による偏微分を表示する機能はついていない.
これらの場合は,後述する Abbreviation のコマンドを用いれば多少楽に打てる.

\end{itemize}
\begin{tabular}{lll}
\hline
たとえば&こんな&ときに\\\hline\hline
このように&分数$\dfrac12$が&詰まる\\\hline
このように&分数$\fpillar\dfrac12$が&詰まらない\\\hline
このように&分数$\dfrac12$が&詰まる\\\hline
\end{tabular} 


\end{document}