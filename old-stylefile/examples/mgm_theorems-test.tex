\documentclass{jsarticle}

%\usepackage[LA,func]{mgmmath}
\usepackage[dependent,section]{mgm_theorems}

\title{mgm\_\,theorems.sty test}
\author{最上伸一}

\begin{document}
\maketitle

mgm\_\,theorems.sty のテスト.
文書そのものの体裁を変えるようなことは特にしない
(ただし,geometry パッケージにて,余白などの体裁は多少変更されている).

\section{用意されている定理環境}
用意されている定理環境は
\begin{itemize}
\item \texttt{defi}:定義 (Definition)
\item \texttt{theo}:定理 (Theorem)
\item \texttt{lemma}:補題 (Lemma)
\item \texttt{corol}:系 (corollary)
\item \texttt{propo}:命題 (proposition)
\item \texttt{exam}:例
\item \texttt{proof}:証明 (Proof)
\item \texttt{shortproof}:略証 (Short proof)
\end{itemize} 
である.カッコ内は英語オプション指定時に表示される名称.

\section{オプションについて}
\begin{defi}[パッケージオプション]
\texttt{\yen usepackage[plain]{mgm\_\,theorems}}のように,
\texttt{usepackage}の直後につける\texttt{plain}などの指定をパッケージオプションという.
\end{defi}
\begin{exam}
\texttt{plain, english, dependent} などはパッケージオプションである.
\end{exam}

\subsection{スタイルオプション}
\begin{theo}[スタイルオプション]
\texttt{plain, standard, fancy} のいずれか1つをオプションに指定すると,
それに応じて定理環境の類の体裁が変わる.何も指定しなければ,standard とみなされる.
\end{theo}
\begin{shortproof}
実際にやってみるのが早かろう.
\end{shortproof}

\begin{theo}[English option]
\texttt{english} オプションを指定すると,定理名が英語表記になる.
何も指定しなければ日本語表記である.
\end{theo}

\begin{lemma}[依存パラメータ指定]
\texttt{dependent} オプションを指定すると,定理環境と定義環境などが連番になる.
\end{lemma}

\begin{theo}[section によるリセット]
\texttt{section} オプションを指定すると,定理番号がセクションごとにリセットされるようになる.
\end{theo}

\begin{corol}[chapter によるリセット]
\texttt{chapter} オプションを指定すると,チャプターごとにリセットされる.
\end{corol}

\section{こんな感じ}
\begin{defi}
bbb
\end{defi}

\begin{theo}[nanntyara]
hettyara
\end{theo}

\begin{proof}
証明はこのように.

行が変わってもほら,この通り.
\end{proof}

\begin{defi}[aaa]
bbb
\end{defi}

\begin{exam}
なんとかかんとか
\item ほにゃほにゃ
\end{exam}

\section{章が変わって}
\begin{defi}
bbb
\end{defi}

\begin{theo}[ピタゴラス]

\end{theo}

\begin{lemma}
補題です
\end{lemma}

\begin{shortproof}[補題]
簡単.
\end{shortproof}

\begin{proof}[ピタゴラスの定理]
三平方の定理より明らかであろう.

三平方の定理は三平方の定理である.
\end{proof}

\begin{corol}
こんな風に
\end{corol}

\begin{propo}
こんな感じの命題
\end{propo}





\end{document}