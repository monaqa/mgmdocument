\documentclass[dvipdfmx]{bxjsreport}

\usepackage{mgmmathtool}
\usepackage{mgmthm}
\usepackage{doctools}
\usepackage{textcomp}
\usepackage{enumitem}

\def\moGue{mo\raise.4ex\hbox{g}ue}

\title{mgm 系列スタイルファイルについて}
\author{\moGue}
\date{\today}

\begin{document}
\maketitle

本ドキュメントでは,新規に作成した\texttt{mgm} 系スタイルファイルの説明を行う.

\chapter{\texttt{mgmmathtool}について}
以下,実装している機能および例を述べる.

\subsection{略記用コマンド}
\begin{description}[style=nextline]
\item[\cs{3}]
\cs{varepsilon}と同じもの($\3$)を出力する.
\item[\cs{4}]
\cs{Delta}と同じもの($\4$)を出力する.
\item[\cs{6}]
\cs{partial}と同じもの($\6$)を出力する.
\item[\cs{7}]
\cs{nabla}と同じもの($\7$)を出力する.
\item[\cs{8}]
\cs{infty}と同じもの($\8$)を出力する.
\item[\cs{w}]
\cs{omega}と同じもの($\w$)を出力する.

\item[\cs{Set}\marg{arg1}\marg{arg2}]
集合の内包的表記を表現する.たとえば
\begin{latexcode}
\begin{align*}
A&=\Set{(x,y)}{x^2+y^2=1}\\
W^{k,p}(a,b)&=
\Set{f\colon(a,b)\to\rset}{\sum_{j=0}^k\int_a^b \abs{\dif[j]{f}{x}}^p dx<\infty}
\end{align*}
\end{latexcode}
と打つことにより,
\begin{align*}
A&=\Set{(x,y)}{x^2+y^2=1}\\
W^{k,p}(a,b)&=\Set{f\colon(a,b)\to\rset}{\sum_{j=0}^k\int_a^b\abs{\dif[j]{f}{x}}^pdx<\infty}
\end{align*}
を得る.上の例の$A$のように,縦に長くならない集合表記にも使えるが,
真価を発揮するのは$W^{k,p}$の例のように,中身が縦に長くなるときである.

\item[\cs{dif}\oarg{order}\marg{numerator}\marg{denominator}]
導関数$\dif{y}{x}$を出力する.オプション引数を付けることにより,$\dif[2]{y}{x}$とできる.
なお,標準ではテキストモードの中でもディスプレイモードで表示される使用になっている
(内部で\cs{dfrac}を用いているため).
スターを付けて\cs{dif}*[\option{2}]\arg{y}\arg{x}とすれば,
$\dif*[2]{y}{x}$のように中のモードに合わせて表示される.

\item[\cs{pd}\oarg{order}\marg{numerator}\marg{denominator}]
編微分係数$\pd{y}{x}$を出力する.オプション引数を付けることにより,
$\pd[2]{y}{x}$とできる.こちらもスターを付けると$\pd*{y}{x}$のように表示できる.
\end{description}

\section{線形代数}
$\vtr[1,2,\mdots,n]\vtr*[1,2,\mdots,n]=\imat<6,6>[n,n]{A}$

\section{関数}

\chapter{\texttt{mgmthm}について}


\end{document}
